\PassOptionsToPackage{unicode=true}{hyperref} % options for packages loaded elsewhere
\PassOptionsToPackage{hyphens}{url}
%
\documentclass[]{article}
\usepackage{lmodern}
\usepackage{amssymb,amsmath}
\usepackage{ifxetex,ifluatex}
\usepackage{fixltx2e} % provides \textsubscript
\ifnum 0\ifxetex 1\fi\ifluatex 1\fi=0 % if pdftex
  \usepackage[T1]{fontenc}
  \usepackage[utf8]{inputenc}
  \usepackage{textcomp} % provides euro and other symbols
\else % if luatex or xelatex
  \usepackage{unicode-math}
  \defaultfontfeatures{Ligatures=TeX,Scale=MatchLowercase}
\fi
% use upquote if available, for straight quotes in verbatim environments
\IfFileExists{upquote.sty}{\usepackage{upquote}}{}
% use microtype if available
\IfFileExists{microtype.sty}{%
\usepackage[]{microtype}
\UseMicrotypeSet[protrusion]{basicmath} % disable protrusion for tt fonts
}{}
\IfFileExists{parskip.sty}{%
\usepackage{parskip}
}{% else
\setlength{\parindent}{0pt}
\setlength{\parskip}{6pt plus 2pt minus 1pt}
}
\usepackage{hyperref}
\hypersetup{
            pdftitle={pseudo-code.md},
            pdfborder={0 0 0},
            breaklinks=true}
\urlstyle{same}  % don't use monospace font for urls
\setlength{\emergencystretch}{3em}  % prevent overfull lines
\providecommand{\tightlist}{%
  \setlength{\itemsep}{0pt}\setlength{\parskip}{0pt}}
\setcounter{secnumdepth}{0}
% Redefines (sub)paragraphs to behave more like sections
\ifx\paragraph\undefined\else
\let\oldparagraph\paragraph
\renewcommand{\paragraph}[1]{\oldparagraph{#1}\mbox{}}
\fi
\ifx\subparagraph\undefined\else
\let\oldsubparagraph\subparagraph
\renewcommand{\subparagraph}[1]{\oldsubparagraph{#1}\mbox{}}
\fi

% set default figure placement to htbp
\makeatletter
\def\fps@figure{htbp}
\makeatother


\title{pseudo-code.md}
\date{}

\begin{document}
\maketitle

\hypertarget{header-n0}{%
\section{Pseudo-Code}\label{header-n0}}

\textbf{Algorithm 1:} Prunning based clustering

\begin{center}\rule{0.5\linewidth}{\linethickness}\end{center}

\textbf{Input: } \emph{A}: the point set;
\emph{n\textbackslash{}}clusters\_: the target number of clusters;

\textbf{Output: } \emph{S}: the tree set;

\textbf{Algorithm: } KMeans: clusters generated by k-means; Delaunay:
performs the triangulation and returns the network; Kruskal: generates
the MST;

\# rough clustering by k-means

\emph{G} ← \emph{KMeans}(\emph{A});

\emph{S} ← ∅, add \emph{Kruskal}( \emph{Delaunay}(\emph{gi}) ) into
\emph{S}, all \emph{gi} ∈ \emph{G};

\textbf{while} length of \emph{S} \textless{}
\emph{n\textbackslash{}}clusters\_ \textbf{do}

find the cluster with the highest STD in \emph{S} as \emph{t0};

prun \emph{t0} into \emph{t1} and \emph{t2}, where the higher STD of
\emph{t1} and \emph{t2} is the lowest case;

\emph{S} ← \emph{S} \textbackslash{} t0; add \emph{t1}, \emph{t2} to
\emph{S};

\textbf{end while;}

\textbf{return} \emph{S};

\begin{center}\rule{0.5\linewidth}{\linethickness}\end{center}

\textbf{Algorithm 2:} Attribute spatial association based sampling

\begin{center}\rule{0.5\linewidth}{\linethickness}\end{center}

\textbf{Input: } \emph{A}: the point set; \emph{r}: radius parameter;
\emph{min\textbackslash{}}r\_: minimum radius;

\textbf{Output: } \emph{S}: sampled set;

\textbf{Algorithm: } Random: randomly returns an element from the set;
KDE: kernel-density estimate; Label: the label of point, processed by
the clustering algorithm; Dist: the Euclidean distance between the given
two points; Entropy: the entropy of the set;

\emph{S} ← ∅; \emph{T} ← ∅;

\textbf{while} \emph{A} ≠ ∅ \textbf{or} \emph{T} ≠ ∅ \textbf{do}

\textbf{if} \emph{T} ≠ ∅ \textbf{then}

\emph{c} ← \emph{Random}(\emph{T}); \emph{T} ← \emph{T} \textbackslash{}
\emph{c}; add \emph{c} into \emph{S};

\textbf{else}

\emph{c} ← \emph{Random}(\emph{A}); \emph{A} ← \emph{A} \textbackslash{}
\emph{c}; add \emph{c} into \emph{S};

\emph{radius} ← \emph{Max}(\emph{r} / \emph{KDE}(\emph{c}),
\emph{min\textbackslash{}}r\_);

\emph{neighbors} ← ∅;

add \emph{Label}(\emph{pi}) into \emph{neighbors}, \emph{Dist}(\emph{c},
\emph{pi}) ≤ \emph{radius}, all \emph{pi} ∈ (\emph{A} + \emph{T});

\emph{radius} ← (\emph{radius} - \emph{min\textbackslash{}}r\emph{) / (1
+ \_Entropy}(\emph{neighbors})) + \emph{min\textbackslash{}}r\_;

\emph{disabled} ← ∅; \emph{active} ← ∅;

add \emph{pi} into disabled, \emph{Dist}(\emph{c}, \emph{pi}) ≤
\emph{radius}, all \emph{pi} ∈ (\emph{A} + \emph{T});

add \emph{pi} into \emph{active}, \emph{radius} \textless{}
\emph{Dist}(\emph{c}, \emph{pi}) ≤ 2 * \emph{radius}, all \emph{pi} ∈
(\emph{A} + \emph{T});

\emph{A} ← \emph{A} \textbackslash{} \emph{pi}, all \emph{pi} ∈
(\emph{disabled} + \emph{active});

\emph{T} ← \emph{T} \textbackslash{} \emph{pi}, all \emph{pi} ∈
(\emph{disabled} + \emph{active});

add \emph{pi} into \emph{T}, all \emph{pi} ∈ \emph{active};

\textbf{end while};

\textbf{return} \emph{S};

\begin{center}\rule{0.5\linewidth}{\linethickness}\end{center}

\end{document}
